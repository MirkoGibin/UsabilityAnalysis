\section{Altre Considerazioni}
In questa sezione si riassumono brevemente alcune considerazioni sul sito in analisi nella sua interezza.
\subsection{Testo}
Poco curata la scelta di colori di sfondo (bianco o immagine) e testo (grigio o bianco), come si può vedere dall'immagine \ref{"Testo e sfondo"} tratte dalla Home Page (\url{https://itstecnologie.it/index.php/ntouch/}).

\IMG{"Testo e sfondo"}
Sarebbe stato preferibile testo nero su sfondo bianco, e fare attenzione a non inserire del testo su una immagine puramente riempitiva al solo scopo di migliorare il design non considerando difficoltà di lettura.\\ 
Altra nota negativa è la dimensione ridotta del font, che rende più difficile la lettura. Non è data la possibilità di resize del testo, costringendo l'utente ad affidarsi agli strumenti del browser.

\subsection{Design}
Non vi sono particolari difetti di "bloated design", se non per qualche immagine riempitiva messa a puro scopo estetico. \\ Si riscontra invece il problema opposto: una disposizione dei contenuti troppo "lasca" e pagine molto lunghe verticalmente che costringono l'utente a fare numerosi scroll e che impedisce la creazione di una mappa mentale efficace. \\
Nota positiva è l'assenza di scroll verticale anche ridimensionando la finestra, dato il layout responsive.

\subsection{Link}
Non viene rispettata la convenzione di sottolineare le parole contenenti link. \\ Non vi è inoltre nessuna convenzione interna (figura \ref{"Link Curiosity"}, \url{https://itstecnologie.it/index.php/curiosity/}): a volte le parole con link non si distinguono dalle altre, altre volte sono colorate di blu. Non viene effettuato il cambio di colore per i link visitati, rendendo più difficile ricordarsi quali pagine si sono già visitate.

\subsection{Ricerche e Pagina 404}
\IMG{Ricerca}

A destra del menu il simbolo di una lente di ingrandimento permette di aprire una funzionalità di ricerca a tutto schermo, priva di filtri. T (figura \ref{Ricerca})ale scelta permette all'utente di eseguire query lunghe e non generare il problema di una casella di testo troppo piccola, ma è un metodo che potrebbe essere non gradito dagli utenti in quanto non abituati. Per annullare la ricerca è necessario cliccare sulla "X" in alto a destra. \\
In alcune pagine viene fornita anche la funzionalità di ricerca classica, con casella di testo e pulsante con lente di ingrandimento (il pulsante con la scritta "Cerca" avrebbe ottenuto un maggior gradimento). Un possibile problema potrebbe essere la casella di testo di soli 17 caratteri, che potrebbe spingere gli utenti ad effettuare ricerche con poche parole chiave e ottenere risultati meno precisi.\\
\\
Qualora la ricerca non avesse successo viene visualizzata la scritta "nessun post trovato" e nessun altro aiuto. Sarebbe stato preferibile guidare l'utente verso altri contenuti.
\IMG{"Error 404"}
L'errore 404 (figura \ref{"Error 404"}) viene gestito specificando che la pagina "non è più qui", senza fornire nessun altro aiuto all'utente. Sarebbe stato interessante fornire dei link ad alcuni contenuti interessanti o nuovi.
