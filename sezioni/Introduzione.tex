\section{Introduzione}

\subsection{Definizione di Usabilità}
L'usabilità è una caratteristica che indica la semplicità d'uso di una cosa, in questo caso un sito internet. Le caratteristiche che la influenzano sono:
\begin{itemize}
	\item La capacità del sito di offrire all'utente ciò che cerca in modo preciso ed accurato;
	\item Quanto tempo (e quanti click) si impiega per raggiungere il proprio obiettivo;
	\item Quanto semplice è imparare ad usare il sito;
	\item Quanto piacevole è l'esperienza di navigazione;
	\item Quanto è alta la possibilità di commettere errori.
\end{itemize}
La forte competizione sul web rende l'usabilità un elemento chiave per distinguersi e per garantire la sopravvivenza del sito. Gli utenti al giorno d'oggi sono abituati a navigare sul web e si abituano a determinati pattern e convenzioni. Data la loro fretta di ottenere ciò che vogliono e la loro poca pazienza, essere troppo originali nella presentazione dei propri contenuti può rivelarsi controproducente, in quanto ciò porterebbe ad uno sforzo eccessivo per adattarsi e comprendere la struttura e le funzionalità offerte. \\
Sono state quindi individuate delle best-practice che permettono di rendere il sito usabile, che verranno utilizzare in questo documento per valutare l'usabilità di un sito internet.

\subsection{Scopo del Documento}
Questo documento ha lo scopo di effettuare l'analisi di usabilità del sito web \url{itstecnologie.com}. Il sito non è particolarmente ampio e di struttura relativamente semplice, fatta eccezione per alcune sezioni del sito nascoste in altre pagine e di raggiungibilità difficoltosa. Verranno analizzate unicamente le pagine di maggior interesse, in quanto le informazioni da esse ricavabili risultano valide per l'intero sito web.
\section{Presentazione del Sito}
\subsection{Contesto}
Il sito in esame è di una piccola azienda che offre servizi relativi a:
\begin{itemize}
	\item Reti idriche e fognarie;
	\item Impianti tecnologici termici, idrici e sanitari;
	\item Arredamento e pavimentazione di case.
\end{itemize}
Il target di clienti è quindi potenzialmente molto vasto: piccoli privati, aziende di varie dimensioni e pubblica amministrazione. \\
Tramite il loro sito si pongono lo scopo di presentare i prodotti di maggior rilievo e delle gallerie di foto dei loro lavori più rappresentativi.

\subsection{Nome del Sito}
Il nome è un elemento importante: è la prima informazione che giunge all'utente. Deve essere breve, semplice e facile da ricordare e scrivere.
"itstecnologie" è relativamente semplice da ricordare, anche se la lettura può essere difficoltosa in un primo momento a causa dell'acronimo "I.T.S.", specialmente per nuovi utenti, per i quali può risultare anche poco chiaro in che contesto si applicano le "tecnologie" riportate nel nome.

\subsection{Struttura}\label{struttura}
Come anticipato, il sito è poco profondo ma ha una struttura non ben curata. A livello superficiale è molto semplice navigare, in quanto dall'Home Page ci si può spostare in varie sezioni raggruppate in un menu:
\begin{itemize}
	\item Servizi;
	\item N-Touch;
	\item Il Palladio;
	\item Portfolio;
	\item Curiosity;
	\item Utility;
	\item Contatti;
\end{itemize}
All'interno di ogni sezione vi sono però altre pagine raggiungibili, contenenti a loro volta dei link ai contenuti veri e propri. La struttura disorganizzata obbliga l'utente a numerosi click per poter arrivare ai contenuti veri e propri, e ciò rende difficile e poco intuitivo il ritrovamento delle informazioni. L'utilizzo di alcune voci in inglese in un sito completamente italiano complicano la situazione.
